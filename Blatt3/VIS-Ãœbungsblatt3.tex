\documentclass[ngerman]{fbi-aufgabenblatt}
	\usepackage{tikz}
	\usepackage{listings}
	\usepackage{color}
	\definecolor{lightgray}{rgb}{.9,.9,.9}
	\definecolor{darkgray}{rgb}{.4,.4,.4}
	\definecolor{purple}{rgb}{0.65, 0.12, 0.82}
	% Folgende Angaben bitte anpassen
	\lstdefinelanguage{JavaScript}{
		keywords={typeof, new, true, false, catch, function, return, null, catch, switch, var, if, in, while, do, else, case, break},
		keywordstyle=\color{blue}\bfseries,
		ndkeywords={class, export, boolean, throw, implements, import, this},
		ndkeywordstyle=\color{darkgray}\bfseries,
		identifierstyle=\color{black},
		sensitive=false,
		comment=[l]{//},
		morecomment=[s]{/*}{*/},
		commentstyle=\color{purple}\ttfamily,
		stringstyle=\color{red}\ttfamily,
		morestring=[b]',
		morestring=[b]"
	}
	\renewcommand{\Vorlesung}{VIS}
	\renewcommand{\Semester}{WiSe 2017}
	\renewcommand{\Aufgabenblatt}{3}
	\renewcommand{\Teilnehmer}{Hennings, Regorz, Röder, Budde, Warrelmann}
	\begin{document}
	\setcounter{section}{0}

	\aufgabe{Backup}

	\subsection*{1)}
	\begin{description}
	\item [Verlust] Der physische Datenträger wird gestohlen oder geht verloren.
	\item [Hardwaredefekte] Der Datenträger erleidet einen physischen, ggf. irreparablen Defekt.
	\item [höhere Gewalt] Der Datenträger wird durch unbeeinflussbare Faktoren beschädigt (bspw. ein Feuer im Haushalt).
	\item [menschliches Versagen] bspw. versehentliches Löschen durch unachtsames Nutzen von \texttt{rm -r /*}
	\item[Malware] Ransomware verschlüsselt die gesamte Festplatte oder Teile dieser.
	\end{description} 
	\subsection*{4)}
	\begin{description}
	\item[Inkrementelles Backup] 
	Sichert alle seit dem letzten Backup hinzugekommenen und veränderten Daten.
	Diese Schritte sind mehrfach hintereinander durchführbar und alle diese Zwischenstände zwischen den Vollbackups sind rekonstruierbar. Der Restore erfolgt in Reihenfolge der Sicherungen und alle Sicherungen seit dem letzten Vollbackup bis einschließlich zum gewünschten Wiederherstellungspunkt sind erforderlich. 
	
	\item [Differenzielles Backup] 
	Der Unterschied zum inkrementellen Backup besteht hauptsächlich darin, dass nur die Daten gesichert werden, die seit dem letzten Voll-Backup hinzugekommenen sind und/oder verändert wurden.
	Der Restore erfordert nur ein Vollbackup und dasjenige differentielle Backup des gewünschten Wiederherstellungspunktes. 
	\end{description} 
	Der Vorteil von inkrementellen Backups gegenüber differenziellen liegt hauptsächlich darin, dass sie idR. weniger Speicherkapazität und Berechnungszeit benötigen. Es handelt sich hierbei schließlich im Grunde nur um ein differenzielles Backup zum letzten inkr. Backup-State, der meistens aktueller als das letzte Voll-Backup sein sollte. Die Vorteile des inkrementellen Backups beim Anlegen und speichern eines neuen Wiederherstellungspunktes stehen dabei jedoch im Gegensatz zu den Nachteilen beim Restore eines solchen States. Es wird die gesamte „Sicherungskette“ aller inkr. Backup-States bis zum gewünschten Wiederherstellungspunkt benötigt. Ist eines dieser Backups inkonsistent oder beschädigt, wird der Restore-Prozess beeinträchtigt oder ist im schlimmsten Fall nicht möglich. Ebenfalls ist der Restore-Prozess ineffizient, da auch solche Dateien erfasst werden, die ggf. zwischen dem Voll-Backup und dem Wiederherstellungspunkt angelegt \& gelöscht wurden, also unerheblich sind. Differentielle Backups sind von diesen Problemen nicht betroffen.
	\aufgabe{RAID}

\subsection*{5)}
Da RAID nicht vor versehentlichem oder bösartigem Löschen schützt, erspart es kein zusätzliches Backup.

\subsection*{7)}
\begin{description}
\item[RAID-0] 2000 GB
\item [RAID-1] 1000 GB 
\item [RAID-4] 1500 GB Nutzdaten / 500 GB Parität
\item [RAID-5] 1500 GB Nutzdaten / 500 GB Parität
\item [RAID-6] 1000 GB
\item[RAID-10] 1000 GB
\end{description} 

\subsection*{8b)}

Mit Small-Write Algorithmus:
\begin{flalign*}
	& 10010111& \\
	\oplus \ &01101011& \\ 
	\oplus \ &00101101& \\
	= \ &11010001&
\end{flalign*}
Ohne Small-Write Algorithmus:
\begin{flalign*}
	& 10010111\\
	\oplus \ &10010011&\\
	\oplus \ &00011011&\\ 
	\oplus \ &11001110&\\
	= \ &11010001&
\end{flalign*}

Neue Belegung:\\\\
HDD-1 : 10010111\\
HDD-P : 11010001\\\\
Der Small-Write Algorithmus muss weniger Lese- und Rechenoperationen für die Paritätsberechnung durchführen. Hierbei entspricht konstant 
\begin{center}
	1 logischer Schreibzugriff = 2 physische Lesezugriffe + 2 physische Schreibzugriffe
\end{center}
Ohne den Algorithmus steigen die Lese- und Rechenoperationen proportional zu der Anzahl eingesetzter Speichermedien für das Einfügen von Dateien, da zur Paritätsberechnung jedes Speichermedium betrachtet und verrechnet werden müsste.

	\subsection*{8c)}
	Das Codewort ist „Plattenredundanz“. Wir haben den folgenden Quelltext in Java implementiert, um die zweite Festplatte wiederherzustellen und das Bild zu laden:
	\lstset{
		language=JavaScript, % nur damit es gut aussieht
		tabsize=2,
		backgroundcolor=\color{lightgray},
		extendedchars=true,
		basicstyle=\footnotesize\ttfamily,
		showstringspaces=false,
		showspaces=false,
		numbers=left,
		numberstyle=\footnotesize,
		numbersep=9pt,
		tabsize=2,
		breaklines=true,
		showtabs=false,
		captionpos=b
	}
	\begin{lstlisting}
	import java.io.FileInputStream;
	import java.io.FileNotFoundException;
	import java.io.FileOutputStream;
	import java.io.IOException;
	import java.io.InputStream;
	import java.io.OutputStream;
	import java.util.Arrays;
	import java.util.List;
	import java.util.function.Function;
	import java.util.stream.Collectors;
	/**
	* Dieses Programm kann ohne Parameter gestartet werden und geht davon
	* aus, dass die Festplatten im Verzeichnis 'data/' abgelegt sind. Es
	* stellt zunaechst die Festplatte 2 in der Datei 'data/disk2Rec'
	* wieder her und laedt die Daten anschliessend in die Datei
	* 'data/image.jpg'
	*/
	public class RecoverAndWrite {
		private static final int CHUNKSIZE = 4096;
		/**
		* Iteriert Block fuer Block ueber die gebenen Festplatten und
		* uebergibt in jeder Iteration einen Block von jeder Festplatte
		* an die Funktion f. Das Ergebnis von f wird in die Datei out
		* geschrieben.
		* 
		* @param ins
		*            Die Pfade zu den Festplatten, von denen gelesen
		*            werden soll
		* @param f
		*            Die Funktion, der ein Block von jeder Festplatte
		*            uebergeben wird
		* @param out
		*            Der Pfad zu der Output-Datei
		*/
		private static void chunkapplyAndWrite(String[] ins,
		Function<List<byte[]>, byte[]> f, String out)
		throws IOException {
			List<InputStream> inputStreams = Arrays.asList(ins).stream()
			.map((in) -> {
				try {
					return new FileInputStream(in);
				} catch (FileNotFoundException e) {
					e.printStackTrace();
					return null;
				}
			}).collect(Collectors.toList());
			OutputStream outputStream = new FileOutputStream(out);
			boolean reading = true;
			while (reading) {
				List<byte[]> chunks = inputStreams.stream()
				.map((inputStream) -> {
					byte[] chunk = new byte[CHUNKSIZE];
					try {
						if (inputStream.read(chunk) == -1) {
							chunk = null;
						}
					} catch (IOException e) {
						e.printStackTrace();
					}
					return chunk;
				}).collect(Collectors.toList());
				reading = !chunks.contains(null);
				if (reading) {
					byte[] appliedChunk = f.apply(chunks);
					outputStream.write(appliedChunk);
				}
			}
			outputStream.close();
		}
		/**
		* Stellt einen Block wieder her.
		*/
		private static byte[] restore(List<byte[]> chunks) {
			byte[] result = new byte[CHUNKSIZE];
			for (byte[] chunk : chunks) {
				for (int i = 0; i < CHUNKSIZE; ++i) {
					result[i] = (byte) (result[i] ^ chunk[i]);
				}
			}
			return result;
		}
		/**
		* Konkateniert die Bloecke, um einen Teil des Bildes zu laden.
		*/
		private static byte[] load(List<byte[]> chunks) {
			byte[] result = new byte[CHUNKSIZE * chunks.size()];
			for (int i = 0; i < chunks.size(); ++i) {
				for (int j = 0; j < CHUNKSIZE; ++j) {
					result[i * CHUNKSIZE + j] = chunks.get(i)[j];
				}
			}
			return result;
		}
		public static void main(String[] args) throws IOException {
			String d0 = "data/disk0";
			String d1 = "data/disk1";
			String d2Rec = "data/disk2Rec";
			String d3 = "data/disk3";
			String img = "data/image.jpg";
			String[] restoreIns = { d0, d1, d3 };
			chunkapplyAndWrite(restoreIns, (chunks) -> restore(chunks),
			d2Rec);
			String[] loadIns = { d0, d1, d2Rec };
			chunkapplyAndWrite(loadIns, (chunks) -> load(chunks), img);
		}
	}
	\end{lstlisting}
\end{document}