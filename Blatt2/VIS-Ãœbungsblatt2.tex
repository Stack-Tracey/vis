\documentclass[ngerman]{fbi-aufgabenblatt}

\usepackage{tikz}

% Folgende Angaben bitte anpassen

\renewcommand{\Vorlesung}{VIS}
\renewcommand{\Semester}{WiSe 2017}

\renewcommand{\Aufgabenblatt}{2}
\renewcommand{\Teilnehmer}{Hennings, Regorz, Röder, Budde, Warrelmann}

\begin{document}
\setcounter{section}{1}
\aufgabe{Deadlocks}

\subsection*{c)}
Die Deadlock-Erkennung in Verteilten Systemen ist wesentlich aufwendiger als in zentralisierten. Es werden Kommunikationsvorgänge zur Mitteilung von Wartebeziehungen benötigt und ein einfacher wait-for Graph ist nicht mehr hinreichend.
 Da jeder Knoten nur für lokale Transaktionen die derzeitige Wartesituation kennt, ist es schwierig eine korrekte Deadlock-Erkennung zu realisieren. Warteinformationen von externen Transaktionen sind wegen ihrer zeitlichen Verzögerungen bei der Übertragung idR. veraltet. 
Eine mögliche Lösung wäre das Obermark-Verfahren
Dieses geht davon aus, dass in jedem Rechner ein Prozess läuft, der sich um die Deadlock-Erkennung kümmert und mit den entsprechenden Prozessen der anderen Teilnehmer kommuniziert. Jeder beteiligte Prozess hat einen lokalen wait-for Graphen mit sämtlichen Wartebeziehungen lokaler und externer Transaktionen. Damit lassen sich lokale Deadlocks sofort und ohne Kommunikation erkennen. 

Im wait-for Graphen wird ein Knoten $EX$ geführt, der Wartebeziehungen zu Sub-Transaktionen $T_{i}$ auf anderen Rechnern markiert. $T_{i}$ bekommt einen global eindeutigen Zeitstempel $tm(T_{i})$ und es wird 
eine Wartebeziehung $EX \rightarrow T_{i}$ aufgenommen, da möglicherweise an anderen Rechnern auf Sperren von $T_{i}$ gewartet wird. Eine Wartebeziehung $T_{i} \rightarrow EX$ wird eingeführt, sobald $T_{i}$ eine externe Sub-Transaktion startet, da diese einen Konflikt erzeugen könnte. Ein potentieller globaler Deadlock wird durch einen Zyklus im lokalen wait-for Graphen angezeigt, an dem der EX-Knoten beteiligt ist:

$EX \rightarrow T_{1} \rightarrow T_{2} \rightarrow ... \rightarrow T_{n} \rightarrow EX$

Um festzustellen, ob tatsächlich ein globaler Deadlock vorliegt, wird die Warteinformation an den Rechner weitergeleitet, an dem Transaktion $T_{i}$ die Sub-Transaktion gestartet hatte. Nach Empfang solcher Warteinformationen vervollständigt der zuständige Prozess seinen wait-for Graphen und führt darauf eine Zyklensuche durch. Zur Erkennung eines globalen Deadlocks kann dazu erneut die Weiterleitung der erweiterten Warteinformation an einen anderen Rechner erforderlich sein. Wird ein vollständiger Zyklus festgestellt, erfolgt die Rücksetzung einer der beteiligten Transaktionen. Dabei sollte, wenn möglich, eine lokale Transaktion ausgewählt werden, um den Deadlock möglichst schnell zu beheben.

Die Weiterleitung der Warteinformation wird nur dann vorgenommen, wenn 
$tm (T_{n}) < tm (T_{1})$ gilt. Damit wird bei einem globalen Deadlock mit zwei Rechnern gewährleistet, dass nur einer die Warteinformation weiterleitet. Bei mehr als zwei Rechnern ist es jedoch weiterhin möglich, dass diese von mehr als einem Rechner weitergegeben wird.


\aufgabe{Threads}

\subsection*{a)}
Die Ausführung eines Programmes wird durch das Betriebssystem in einem Prozess gekapselt. Der Prozess erhält hierbei einen eigenen Adressraum im Speicher, sodass kein Prozess die Daten eines anderen Prozesses lesen kann. Der Informationsaustausch zwischen zwei Prozessen kann über Nachrichten realisiert werden. Die nebenläufigen Teile eines Programms werden als Threads ausgeführt, die auf den Speicher des Prozesses zugreifen können. Alle Threads eines Prozesses haben also einen gemeinsamen Speicher.

\subsection*{c)}
Das Problem bei der Verwendung der \textit{sleep}-Methode eines Java-Threads zur Lösung von Nebenläufigkeitsproblemen, ist, dass das Verwenden der \textit{sleep}-Methode nicht sicherstellt, dass die Ausführungsreihenfolge, die durch die Verwendung erzwungen werden soll, wirklich eingehalten wird. \\
\\
Das Problem entsteht dadurch, dass nicht festgelegt ist, wann welcher Teil eine Threads zur Ausführung kommt, und die Reihenfolge bei mehrfacher Ausführung variieren kann. Hierdurch kann es sein, dass ein Thread durch die \textit{sleep}-Methode dazu gezwungen werden soll, auf die Ausführung einer Aktion eines anderen Threads für beispielsweise $2$ Sekunden zu warten, dieser andere Thread in diesen $2$ Sekunden aber nicht zur Ausführung kommt. Hierdurch kann dazu kommen, dass die Reihenfolge trotz der $2$ Sekunden falsch ist.
\subsection*{d)}

\aufgabe{Nebenläufigkeit}

\subsection*{a)}
\begin{tikzpicture}[->]
\node (Top) at (0,0) {};
\node[below] (Bottom) at (0,-7) {Zeit};
\node[right] (HT) at (1,0.5) {Hauptthread};
\node[right] (DT) at (5, 0.5) {Downloadthread};
\draw (Top) -- (Bottom);

\draw (0.5,0) rectangle (4,-1);
\node (Drei) at (2.25,-0.5) {3};
\draw (0.5,-1) rectangle (4,-2);
\node (Zwei) at (2.25, -1.5) {2};
\draw (5,-2) rectangle (8.5,-5);
\node (Eins) at (6.75,-3.5) {1};
\draw (0.5,-5) rectangle (4,-6);
\node (Vier) at (2.25,-5.5) {4};
\draw (0.5,-6) rectangle (4,-7);
\node (Fuenf) at (2.25, -6.5) {5};

\end{tikzpicture}
\subsection*{b)}
Wenn der Ablauf so ausgeführt wird, wie er in der Zeichnung aus Aufgabenteil a) zu sehen ist, wird das heruntergeladene Bild angezeigt, da der Hauptthread bei dieser Ausführungsreihenfolge erst nach dem beenden des Downloadthreads fortfährt. Das Problem des Quellcodes ist, dass er nicht festlegt, dass der Hauptthread auf den Downloadthread warten muss. Es wäre also auch die folgende Reihenfolge möglich, wenn der Hauptthread nach dem Start des Downloadthreads parallel zum Downloadthread weiterläuft: \\

\begin{tikzpicture}[->]
\node (Top) at (0,0) {};
\node[below] (Bottom) at (0,-5) {Zeit};
\node[right] (HT) at (1,0.5) {Hauptthread};
\node[right] (DT) at (5, 0.5) {Downloadthread};
\draw (Top) -- (Bottom);

\draw (0.5,0) rectangle (4,-1);
\node (Drei) at (2.25,-0.5) {3};
\draw (0.5,-1) rectangle (4,-2);
\node (Zwei) at (2.25, -1.5) {2};
\draw (5,-2) rectangle (8.5,-5);
\node (Eins) at (6.75,-3.5) {1};
\draw (0.5,-2.5) rectangle (4,-3.5);
\node (Vier) at (2.25,-3) {4};
\draw (0.5,-3.5) rectangle (4,-4.5);
\node (Fuenf) at (2.25, -4) {5};

\end{tikzpicture} \\
Da die Zuweisung zum \textit{result}-Feld die letzte Aktion des Ausführungsschritts 1 ist, wird der Ausführungsschritt 4 nicht das anzuzeigende Bild von dem Downloader-Objekt bekommen, sodass es in Ausführungsschritt 5 nicht angezeigt wird.
\subsection*{c)}
Die \textit{join()}-Methode kann auf einem Thread-Objekt aufgerufen werden, um auf das Beenden des Threads zu warten. Die Ausführung des aufrufenden Threads wird durch den Aufruf pausiert, bis der andere Thread beendet ist.

\end{document}
