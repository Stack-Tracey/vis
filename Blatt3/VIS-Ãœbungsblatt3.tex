\documentclass[ngerman]{fbi-aufgabenblatt}

\usepackage{tikz}
\usepackage{listings}
\usepackage{amsmath}
% Folgende Angaben bitte anpassen

\renewcommand{\Vorlesung}{VIS}
\renewcommand{\Semester}{WiSe 2017}

\renewcommand{\Aufgabenblatt}{3}
\renewcommand{\Teilnehmer}{Hennings, Regorz, Röder, Budde, Warrelmann}

\begin{document}
\setcounter{section}{0}
   

\aufgabe{Backup}

\subsection*{1)}
\begin{description}
	\item[Diebstahl] Der Laptop wird entwendet.
	\item [Hardwaredefekte] Der Laptop läuft voll Wasser oder fällt runter.
	\item [höhere Gewalt] Das Haus brennt und der Laptop steht auf dem Schreibtisch.
	\item [versehentliches Löschen] rm -r /*
	\item[Viren u. andere Schadprogramme] Virus verschlüsselt die gesammte Festplatte.
\end{description} 

\subsection*{4)}
\begin{description}
	\item[Inkerementelles Backup] 
	Sichert alle seit dem letzten Backup hinzugekommenen und veränderten Daten.
	Diese Schritte sind mehrfach hintereinander durchführbar und alle diese Zwischenstände zwischen den Vollbackups sind rekonstruierbar. Der Restore erfolgt in Reihenfolge der Sicherungen und alle Sicherungen seit dem letzten Vollbackup sind erforderlich. 
	
	\item [Differenzielles Backup] 
	Der Unterschied zum inkrementellen Backup besteht hauptsächlich darin, dass nur die Daten gesichert werden die seit dem letzten Voll-Backup hinzugekommenen sind und/oder verändert wurden.
	Der Restore erfordert nur ein Vollbackup und das letzte differentielle Backup. 
\end{description} 

Der Vorteil von inkrementellen Backups gegenüber differenziellen liegt hauptsächlich darin, dass sie regelmässig durchgeführt, weniger Speicherkapazität braucht. Auch die Wiederherstellung ab einem präziseren Zeitpunkt ist möglich,  dafür braucht diese jedoch länger und die Handhabung ist komplizierter als bei einem differenziellen Backup, da alle Dateien der „Sicherungskette“ für eine Wiederherstellung benötigt werden.


\aufgabe{RAID}

\subsection*{5)}
Da RAID nicht vor versehentlichem oder bösartigem Löschen schützt, erspart es kein zusätzliches Backup.

\subsection*{7)}
\begin{description}
	\item[RAID-0] 2000 GB
	\item [RAID-1] 500 GB
	\item [RAID-4] 1500 GB Nutzdaten / 500 GB Parität
	\item [RAID-5] 1500 GB Nutzdaten / 500 GB Parität
	\item [RAID-6] 1000 GB
	\item[RAID-10] 1000 GB
\end{description} 

\subsection*{8b)}

Mit Small-Write Algorithmus:\\\\
$.\quad 10010111 \\\oplus \; 01101011 \\ \oplus \; 00101101\\ = \; 11010001$\\\\
Ohne Small-Write Algorithmus:\\\\
$.\quad 10010111 \\\oplus \; 10010011 \\\oplus \; 00011011\\ \oplus \; 11001110 \\= \; 11010001$\\

Neue Belegung:\\\\
$HDD-1 : 10010111$\\
$HDD-P : 11010001$\\\\
Der Small-Write Algorithmus muss weniger Rechenoperationen für die Paritätsberechnungen durchführen. Hierbei sind 1 Logischer Schreibzugriff = 2 physische Lesezugriffe + 2 physische Schreibzugriffe und das konstant. Ohne den Algorithmus steigen die Rechenoperationen proportional zu der Anzahl eingesetzter Speichermedien für das Einfügen von Dateien, da zur Paritätsberechnung jedes Speichermedium betrachtet werden muss.

\subsection*{8c)}
Das Codewort ist 'Plattenredundanz'. Wir haben den folgenden Quelltext implementiert, um die zweite Festplatte wiederherzustellen und das Bild zu laden:
\lstset{tabsize=2}
\begin{lstlisting}
import java.io.FileInputStream;
import java.io.FileNotFoundException;
import java.io.FileOutputStream;
import java.io.IOException;
import java.io.InputStream;
import java.io.OutputStream;
import java.util.Arrays;
import java.util.List;
import java.util.function.Function;
import java.util.stream.Collectors;

/**
 * Dieses Programm kann ohne Parameter gestartet werden und geht davon
 * aus, dass die Festplatten im Verzeichnis 'data/' abgelegt sind. Es
 * stellt zunaechst die Festplatte 2 in der Datei 'data/disk2Rec'
 * wieder her und laedt die Daten anschliessend in die Datei
 * 'data/image.jpg'
 */
public class RecoverAndWrite {

	private static final int CHUNKSIZE = 4096;

	/**
	 * Iteriert Block fuer Block ueber die gebenen Festplatten und
	 * uebergibt in jeder Iteration einen Block von jeder Festplatte
	 * an die Funktion f. Das Ergebnis von f wird in die Datei out
	 * geschrieben.
	 * 
	 * @param ins
	 *            Die Pfade zu den Festplatten, von denen gelesen
	 *            werden soll
	 * @param f
	 *            Die Funktion, der ein Block von jeder Festplatte
	 *            uebergeben wird
	 * @param out
	 *            Der Pfad zu der Output-Datei
	 */
	private static void chunkapplyAndWrite(String[] ins,
			Function<List<byte[]>, byte[]> f, String out)
			throws IOException {
		List<InputStream> inputStreams = Arrays.asList(ins).stream()
				.map((in) -> {
					try {
						return new FileInputStream(in);
					} catch (FileNotFoundException e) {
						e.printStackTrace();
						return null;
					}
				}).collect(Collectors.toList());
		OutputStream outputStream = new FileOutputStream(out);
		boolean reading = true;
		while (reading) {
			List<byte[]> chunks = inputStreams.stream()
					.map((inputStream) -> {
						byte[] chunk = new byte[CHUNKSIZE];
						try {
							if (inputStream.read(chunk) == -1) {
								chunk = null;
							}
						} catch (IOException e) {
							e.printStackTrace();
						}
						return chunk;
					}).collect(Collectors.toList());
			reading = !chunks.contains(null);
			if (reading) {
				byte[] appliedChunk = f.apply(chunks);
				outputStream.write(appliedChunk);
			}
		}
		outputStream.close();
	}

	/**
	 * Stellt einen Block wieder her.
	 */
	private static byte[] restore(List<byte[]> chunks) {
		byte[] result = new byte[CHUNKSIZE];
		for (byte[] chunk : chunks) {
			for (int i = 0; i < CHUNKSIZE; ++i) {
				result[i] = (byte) (result[i] ^ chunk[i]);
			}
		}
		return result;
	}

	/**
	 * Konkateniert die Bloecke, um einen Teil des Bildes zu laden.
	 */
	private static byte[] load(List<byte[]> chunks) {
		byte[] result = new byte[CHUNKSIZE * chunks.size()];
		for (int i = 0; i < chunks.size(); ++i) {
			for (int j = 0; j < CHUNKSIZE; ++j) {
				result[i * CHUNKSIZE + j] = chunks.get(i)[j];
			}
		}
		return result;
	}

	public static void main(String[] args) throws IOException {
		String d0 = "data/disk0";
		String d1 = "data/disk1";
		String d2Rec = "data/disk2Rec";
		String d3 = "data/disk3";
		String img = "data/image.jpg";
		String[] restoreIns = { d0, d1, d3 };
		chunkapplyAndWrite(restoreIns, (chunks) -> restore(chunks),
				d2Rec);
		String[] loadIns = { d0, d1, d2Rec };
		chunkapplyAndWrite(loadIns, (chunks) -> load(chunks), img);
	}
}

\end{lstlisting}

\end{document}
=======
\documentclass[ngerman]{fbi-aufgabenblatt}

\usepackage{tikz}
\usepackage{listings}
\usepackage{color}
\definecolor{lightgray}{rgb}{.9,.9,.9}
\definecolor{darkgray}{rgb}{.4,.4,.4}
\definecolor{purple}{rgb}{0.65, 0.12, 0.82}
% Folgende Angaben bitte anpassen

\lstdefinelanguage{JavaScript}{
  keywords={typeof, new, true, false, catch, function, return, null, catch, switch, var, if, in, while, do, else, case, break},
  keywordstyle=\color{blue}\bfseries,
  ndkeywords={class, export, boolean, throw, implements, import, this},
  ndkeywordstyle=\color{darkgray}\bfseries,
  identifierstyle=\color{black},
  sensitive=false,
  comment=[l]{//},
  morecomment=[s]{/*}{*/},
  commentstyle=\color{purple}\ttfamily,
  stringstyle=\color{red}\ttfamily,
  morestring=[b]',
  morestring=[b]"
}

\renewcommand{\Vorlesung}{VIS}
\renewcommand{\Semester}{WiSe 2017}

\renewcommand{\Aufgabenblatt}{3}
\renewcommand{\Teilnehmer}{Hennings, Regorz, Röder, Budde, Warrelmann}

\begin{document}
\setcounter{section}{0}
   

\aufgabe{Backup}

\subsection*{1)}
\begin{description}
	\item[Diebstahl] Der Laptop wird entwendet.
	\item [Hardwaredefekte] Der Laptop läuft voll Wasser oder fällt runter.
	\item [höhere Gewalt] Das Haus brennt und der Laptop steht auf dem Schreibtisch.
	\item [versehentliches Löschen] rm -r /*
	\item[Viren u. andere Schadprogramme] Virus verschlüsselt die gesammte Festplatte.
\end{description} 

\subsection*{4)}
\begin{description}
	\item[Inkerementelles Backup] 
	Sichert alle seit dem letzten Backup hinzugekommenen und veränderten Daten.
	Diese Schritte sind mehrfach hintereinander durchführbar und alle diese Zwischenstände zwischen den Vollbackups sind rekonstruierbar. Der Restore erfolgt in Reihenfolge der Sicherungen und alle Sicherungen seit dem letzten Vollbackup sind erforderlich. 
	
	\item [Differenzielles Backup] 
	Der Unterschied zum inkrementellen Backup besteht hauptsächlich darin, dass nur die Daten gesichert werden die seit dem letzten Voll-Backup hinzugekommenen sind und/oder verändert wurden.
	Der Restore erfordert nur ein Vollbackup und das letzte differentielle Backup. 
\end{description} 

Der Vorteil von inkrementellen Backups gegenüber differenziellen liegt hauptsächlich darin, dass sie regelmässig durchgeführt, weniger Speicherkapazität braucht. Auch die Wiederherstellung ab einem präziseren Zeitpunkt ist möglich,  dafür braucht diese jedoch länger und die Handhabung ist komplizierter als bei einem differenziellen Backup, da alle Dateien der „Sicherungskette“ für eine Wiederherstellung benötigt werden.


\aufgabe{RAID}

\subsection*{5)}
Da RAID nicht vor versehentlichem oder bösartigem Löschen schützt, erspart es kein zusätzliches Backup.
\newline
Bei RAID-0 ist es ganz klar, dass ein zusätzliches Backup notwendig ist. Bei RAID-1 könnte man im entfernteren Sinne
durch die Spiegelung von einem Backup sprechen, jedoch wird hier auch das versehentliche oder bösartige Löschen auf
die gespielte Platte übertragen. 

\subsection*{7)}
\begin{description}
	\item[RAID-0] 2000 GB
	\item [RAID-1] 500 GB
	\item [RAID-4] 1500 GB Nutzdaten / 500 GB Parität
	\item [RAID-5] 1500 GB Nutzdaten / 500 GB Parität
	\item [RAID-6] 1000 GB
	\item[RAID-10] 1000 GB
\end{description} 

\subsection*{8b)}

\subsection*{8c)}
Das Codewort ist 'Plattenredundanz'. Wir haben den folgenden Quelltext implementiert, um die zweite Festplatte wiederherzustellen und das Bild zu laden:

\lstset{
   language=JavaScript, % nur damit es gut aussieht
   tabsize=2,
   backgroundcolor=\color{lightgray},
   extendedchars=true,
   basicstyle=\footnotesize\ttfamily,
   showstringspaces=false,
   showspaces=false,
   numbers=left,
   numberstyle=\footnotesize,
   numbersep=9pt,
   tabsize=2,
   breaklines=true,
   showtabs=false,
   captionpos=b
}
\begin{lstlisting}
import java.io.FileInputStream;
import java.io.FileNotFoundException;
import java.io.FileOutputStream;
import java.io.IOException;
import java.io.InputStream;
import java.io.OutputStream;
import java.util.Arrays;
import java.util.List;
import java.util.function.Function;
import java.util.stream.Collectors;

/**
 * Dieses Programm kann ohne Parameter gestartet werden und geht davon
 * aus, dass die Festplatten im Verzeichnis 'data/' abgelegt sind. Es
 * stellt zunaechst die Festplatte 2 in der Datei 'data/disk2Rec'
 * wieder her und laedt die Daten anschliessend in die Datei
 * 'data/image.jpg'
 */
public class RecoverAndWrite {

	private static final int CHUNKSIZE = 4096;

	/**
	 * Iteriert Block fuer Block ueber die gebenen Festplatten und
	 * uebergibt in jeder Iteration einen Block von jeder Festplatte
	 * an die Funktion f. Das Ergebnis von f wird in die Datei out
	 * geschrieben.
	 * 
	 * @param ins
	 *            Die Pfade zu den Festplatten, von denen gelesen
	 *            werden soll
	 * @param f
	 *            Die Funktion, der ein Block von jeder Festplatte
	 *            uebergeben wird
	 * @param out
	 *            Der Pfad zu der Output-Datei
	 */
	private static void chunkapplyAndWrite(String[] ins,
			Function<List<byte[]>, byte[]> f, String out)
			throws IOException {
		List<InputStream> inputStreams = Arrays.asList(ins).stream()
				.map((in) -> {
					try {
						return new FileInputStream(in);
					} catch (FileNotFoundException e) {
						e.printStackTrace();
						return null;
					}
				}).collect(Collectors.toList());
		OutputStream outputStream = new FileOutputStream(out);
		boolean reading = true;
		while (reading) {
			List<byte[]> chunks = inputStreams.stream()
					.map((inputStream) -> {
						byte[] chunk = new byte[CHUNKSIZE];
						try {
							if (inputStream.read(chunk) == -1) {
								chunk = null;
							}
						} catch (IOException e) {
							e.printStackTrace();
						}
						return chunk;
					}).collect(Collectors.toList());
			reading = !chunks.contains(null);
			if (reading) {
				byte[] appliedChunk = f.apply(chunks);
				outputStream.write(appliedChunk);
			}
		}
		outputStream.close();
	}

	/**
	 * Stellt einen Block wieder her.
	 */
	private static byte[] restore(List<byte[]> chunks) {
		byte[] result = new byte[CHUNKSIZE];
		for (byte[] chunk : chunks) {
			for (int i = 0; i < CHUNKSIZE; ++i) {
				result[i] = (byte) (result[i] ^ chunk[i]);
			}
		}
		return result;
	}

	/**
	 * Konkateniert die Bloecke, um einen Teil des Bildes zu laden.
	 */
	private static byte[] load(List<byte[]> chunks) {
		byte[] result = new byte[CHUNKSIZE * chunks.size()];
		for (int i = 0; i < chunks.size(); ++i) {
			for (int j = 0; j < CHUNKSIZE; ++j) {
				result[i * CHUNKSIZE + j] = chunks.get(i)[j];
			}
		}
		return result;
	}

	public static void main(String[] args) throws IOException {
		String d0 = "data/disk0";
		String d1 = "data/disk1";
		String d2Rec = "data/disk2Rec";
		String d3 = "data/disk3";
		String img = "data/image.jpg";
		String[] restoreIns = { d0, d1, d3 };
		chunkapplyAndWrite(restoreIns, (chunks) -> restore(chunks),
				d2Rec);
		String[] loadIns = { d0, d1, d2Rec };
		chunkapplyAndWrite(loadIns, (chunks) -> load(chunks), img);
	}
}

\end{lstlisting}

\end{document}
>>>>>>> 08270f603f00a2a91e7c9eb2d2ce5e813687d1bd
