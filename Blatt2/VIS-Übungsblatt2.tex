\documentclass[ngerman]{fbi-aufgabenblatt}

% Folgende Angaben bitte anpassen

\renewcommand{\Vorlesung}{VIS}
\renewcommand{\Semester}{WiSe 2017}

\renewcommand{\Aufgabenblatt}{2}
\renewcommand{\Teilnehmer}{Hennings, Regorz, Röder, Budde, Warrelmann}

\begin{document}
\setcounter{section}{1}
\aufgabe{Deadlocks}

\subsection*{c)}


\aufgabe{Threads}

\subsection*{a)}
Die Ausführung eines Programmes wird durch das Betriebssystem in einem Prozess gekapselt. Der Prozess erhält hierbei einen eigenen Adressraum im Speicher, sodass kein Prozess die Daten eines anderen Prozesses lesen kann. Der Informationsaustausch zwischen zwei Prozessen kann über Nachrichten realisiert werden. Die nebenläufigen Teile eines Programms werden als Threads ausgeführt, die auf den Speicher des Prozesses zugreifen können. Alle Threads eines Prozesses haben also einen gemeinsamen Speicher.

\subsection*{c)}
\subsection*{d)}

\aufgabe{Nebenläufigkeit}

\subsection*{a)}
\subsection*{b)}
\subsection*{c)}

\end{document}
