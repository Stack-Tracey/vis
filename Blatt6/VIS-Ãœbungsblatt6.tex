\documentclass[ngerman]{fbi-aufgabenblatt}
%	\usepackage{tikz}
	\usepackage{listings}
	\usepackage{color}
	\usepackage{amsmath}
	\definecolor{lightgray}{rgb}{.9,.9,.9}
	\definecolor{darkgray}{rgb}{.4,.4,.4}
	\definecolor{purple}{rgb}{0.65, 0.12, 0.82}
	% Folgende Angaben bitte anpassen
	\renewcommand{\Vorlesung}{VIS}
	\renewcommand{\Semester}{WiSe 2017}
	\renewcommand{\Aufgabenblatt}{6}
	\renewcommand{\Teilnehmer}{Hennings, Regorz, Röder, Budde, Warrelmann}
	\begin{document}
	\setcounter{section}{0}

	\aufgabe{Logische Uhren}
	
	\subsection*{a)}
	\begin{tabular}{c|c|c}
		Ereignis & Lamportzeit & Vektorzeit \\
		\hline
		a & 3 & (1,0,2)\\
		b & 4 & (2,0,2) \\
		c & 9 & (3,5,3) \\
		
		d & 2 & (0,1,1) \\
		e & 5 & (2,2,2) \\
		f & 6 & (2,3,2) \\
		g & 7 & (2,4,3) \\
		h & 8 & (2,5,3) \\
		
		i & 1 & (0,0,1) \\
		j & 2 & (0,0,2) \\
		k & 3 & (0,0,3) \\
		l & 7 & (2,3,4) \\
	\end{tabular}

	\subsection*{b)}
	
    \aufgabe{Anpassung von Zeit}
    
	\aufgabe{Zeitsynchronisation}
	
	\subsection*{a)}
	Die im Folgenden genutzten Variablen sind wie auf der Folie über die Cristians-Methode aus der Vorlesung benannt.
	\begin{align*}
		t&=10:14:05\\
		T_{round} &= 10:14:06 - 10:14:00 = 00:00:06 \\
		T_{trans} &= \frac{T_{round}}{2}= \frac{00:00:06}{2}=00:00:03 \\
		T_{sync} &= t+T_{trans} = 10:14:05+00:00:03 = 10:14:08
	\end{align*}
	\subsection*{b)}
	Die Berechnung mit der Cristians-Methode basiert auf der Annahme, dass die beiden Nachrichten die gleiche Laufzeit haben. Diese Annahme wird benötigt, um die Übertragungszeit der Antwort aus der Summe der Übertragungszeiten errechnen zu können.
	\aufgabe{Bully-Algorithmus}
	
	\subsection*{a)}
	Der Bully-Algorithmus wird eingesetzt, sobald ein Koordinator Prozess in einem verteilten System abstürzt und ein neuer benötigt wird.
	Wenn ein Prozess p z.B. durch einen timeout feststellt, dass der Koordinator Prozess x nicht mehr erreichbar ist, sendet dieser einen request an alle Prozesse mit höherer ID Q, so dass gilt Q.ID > p.ID und wartet auf deren response. Kommt einer, hört p auf zu senden und der Prozess der geantwortet hat, sendet wiederum rekursiv an alle höheren Prozesse. Es gilt p = x sobald p keine Antworten erhält. Dieser teilt per boradcast mit nun neuer Koodinator zu sein.

	5 sendet request an 9 und 11\\
	5 bekommt Antwort von 9\\
	5 hört auf zu senden\\
	9 sendet request an 11\\
	9 bekommt keine Antwort\\
	9 wird Koordinator\\
	9 broadcasted die Nachricht der neue Koodinator zu sein an 3 und 5 	
		
	\end{document}
