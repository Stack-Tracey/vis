\documentclass[ngerman]{fbi-aufgabenblatt}

% Folgende Angaben bitte anpassen

\renewcommand{\Vorlesung}{VIS}
\renewcommand{\Semester}{WiSe 2017}

\renewcommand{\Aufgabenblatt}{2}
\renewcommand{\Teilnehmer}{Hennings, Regorz, Röder, Budde, Warrelmann}

\begin{document}
\setcounter{section}{1}
\aufgabe{Deadlocks}

\subsection*{c)}


\aufgabe{Threads}

\subsection*{a)}
Die Ausführung eines Programmes wird durch das Betriebssystem in einem Prozess gekapselt. Der Prozess erhält hierbei einen eigenen Adressraum im Speicher, sodass kein Prozess die Daten eines anderen Prozesses lesen kann. Der Informationsaustausch zwischen zwei Prozessen kann über Nachrichten realisiert werden. Die nebenläufigen Teile eines Programms werden als Threads ausgeführt, die auf den Speicher des Prozesses zugreifen können. Alle Threads eines Prozesses haben also einen gemeinsamen Speicher.

\subsection*{c)}
Das Problem bei der Verwendung der \textit{sleep}-Methode eines Java-Threads zur Lösung von Nebenläufigkeitsproblemen, ist, dass das Verwenden der \textit{sleep}-Methode nicht sicherstellt, dass die Ausführungsreihenfolge, die durch die Verwendung erzwungen werden soll, wirklich eingehalten wird. \\
\\
Das Problem entsteht dadurch, dass nicht festgelegt ist, wann welcher Teil eine Threads zur Ausführung kommt, und die Reihenfolge bei mehrfacher Ausführung variieren kann. Hierdurch kann es sein, dass ein Thread durch die \textit{sleep}-Methode dazu gezwungen werden soll, auf die Ausführung einer Aktion eines anderen Threads für beispielsweise $2$ Sekunden zu warten, dieser andere Thread in diesen $2$ Sekunden aber nicht zur Ausführung kommt. Hierdurch kann dazu kommen, dass die Reihenfolge trotz der $2$ Sekunden falsch ist.
\subsection*{d)}

\aufgabe{Nebenläufigkeit}

\subsection*{a)}
\subsection*{b)}
\subsection*{c)}

\end{document}
