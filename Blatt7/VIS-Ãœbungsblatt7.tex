\documentclass[ngerman]{fbi-aufgabenblatt}
%	\usepackage{tikz}
	\usepackage{listings}
	\usepackage{color}
	\usepackage{amsmath}
	\definecolor{lightgray}{rgb}{.9,.9,.9}
	\definecolor{darkgray}{rgb}{.4,.4,.4}
	\definecolor{purple}{rgb}{0.65, 0.12, 0.82}
	% Folgende Angaben bitte anpassen
	\renewcommand{\Vorlesung}{VIS}
	\renewcommand{\Semester}{WiSe 2017}
	\renewcommand{\Aufgabenblatt}{7}
	\renewcommand{\Teilnehmer}{Hennings, Regorz, Röder, Budde, Warrelmann}
	\begin{document}
	\setcounter{section}{0}

	\aufgabe{Kryptografie}
	
	\subsection*{a)}
	Der Angreifer könnte relativ einfach mit einer known-plaintext-attack die Schlüssel $K_{1}$ und $K_{2}$ heraus finden.
	Hierbei wird der bekannte Klartext M mit allen möglichen Belegungen für $K_{1}$ verschlüsselt. Beide Werte für $K_{1}$ und v werden in einer Tabelle gespeichert $[[\{value: k_{1i}\} ,\{v_{i}\}], ... ]$ Das selbe geschieht für die bekannte Chiffre C aus M. Diese wird wiederum mit allen möglichen Belegungen für $K_{2}$ entschlüsselt und auch hier werden die Werte in eine Tabelle geschrieben $[[\{value: k_{2j}\} ,\{w_{j}\}], ... ]$ . Gillt nun $v_{i} == w_{j}$ für ein bestimmtes Paar i und j, dann sind $K_{1i}$ und $K_{2j}$ die gesuchten Schlüssel.\\\
	Der Auffwand beträgt:
	$2^{56}+2^{56} =2\cdot 2^{56} =2^{57}$\\\
	–  Sicherheitsgewinn wäre nur 1 Bit
	\\\\\\
	 \textbf{Wir sollen diese Sicherheit mit der Vergleichen, die durch zwei 56 bit Ks erreicht wird, aber genau das tun wir ja hier. Weis jemand von euch was die wollen??}
		
	\subsection*{b)}
	Die effektive Sicherheit eines 3-DES mit 3 Schlüsseln:\\\
	$2^{56}+2^{56}+2^{56} =2\cdot 2\cdot 2^{56} =2^{58}$\\\
	–  Sicherheitsgewinn wären somit zwei Bit, was im Gegensatz zum Sicherheitsgewinn von DES zu DES-2 signifikant mehr ist, da es sich hier um exponentielles Wachstum handelt. Der 3-DES mit 3 Schlüsseln ist doppelt so sicher wie
	 2-DES.
	 	\\\\\\
	 \textbf{Der Aufgabenteil ist nur angefangen und mit Unsicherheiten, kann sich da noch wer dran setzen?}
		
    \aufgabe{Unsicherheit des Electronic-Codebook-Modus}
    
	\aufgabe{Cipher-Block-Chaining-Modus}
	
	\subsection*{a)}
	
	\subsection*{b)}
	
	\subsection*{c)}

	\subsection*{d)}
	
	\aufgabe{Hybrides Kryptosystem}
	
	\aufgabe{Das Diffi-Hellmann-Schlüsselaustauschprotokoll}
	
	\aufgabe{Sicherheit des RSA Verfahrens}
	
	\subsection*{a)}
	
	\subsection*{b)}
	
		
	\end{document}
