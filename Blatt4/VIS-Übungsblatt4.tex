\documentclass[ngerman]{fbi-aufgabenblatt}
%	\usepackage{tikz}
	\usepackage{listings}
	\usepackage{color}
	\definecolor{lightgray}{rgb}{.9,.9,.9}
	\definecolor{darkgray}{rgb}{.4,.4,.4}
	\definecolor{purple}{rgb}{0.65, 0.12, 0.82}
	% Folgende Angaben bitte anpassen
	\renewcommand{\Vorlesung}{VIS}
	\renewcommand{\Semester}{WiSe 2017}
	\renewcommand{\Aufgabenblatt}{4}
	\renewcommand{\Teilnehmer}{Hennings, Regorz, Röder, Budde, Warrelmann}
	\begin{document}
	\setcounter{section}{0}

	\aufgabe{Fehlertoleranz}

	\subsection*{3)}
  Die beiden Datenbankserver sind zu 0,5\% nicht verfügbar. Die 150 Webserver
	sind mit jeweils 0,5\% unverfügbar.
	Für den Gesamtausfall ergibt sich folgende Rechnung: \newline
	$0.1* 0.05 + {0.05}^{150} = 0.005 + 0 = 0.005$ \newline
	Da es sich hier um Webserver und Datenbankserver handelt, müssen einerseits
	beide Datenbanken
	oder alle Webserver ausfallen, sodass der Dienst nicht mehr verfügbar ist.
	Da die Ausfallwahrscheinlichkeit der 150 Webserver ganz nah bei 0 liegt, ist
	nur die Ausfallwahrscheinlichkeit der Datenbankserver zu betrachten.
	\newline
	Die Anzahl an Stunden, die der Dienst nicht zur Verfügung steht , ergibt sich wie folgt: \newline
	$365 * 24 * 0.005 = 43.8 Stunden$ \newline
	Der Server ist nach der Rechnung 43.8 Stunden im Jahr nicht verfügbar.
	Das Versprechen ist eine Lüge!
	
	\subsection*{5)}

  \subsubsection*{a}
  Die Gesamtverfügbarkeit ergibt sich wie folgt: \newline
  \newline
  $A = \frac{365 Tage * 5 Jahre * 24 Stunden * 60 Minuten - 260 Minuten}{365 Tage * 5
  Jahre * 24 Stunden * 60 Minuten}= 99.990106 \%$
  

	\aufgabe{Kryptographie}
	
	\subsection*{7)}
  \subsubsection*{a}
  Der Angreifer erfährt die Länge des Klartextes, da es sich hierbei um eine bijektive Abbildung handelt.
    \subsubsection*{b}
      \subsubsection*{c}

\end{document}
