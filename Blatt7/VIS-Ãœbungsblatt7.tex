\documentclass[ngerman]{fbi-aufgabenblatt}
%	\usepackage{tikz}
	\usepackage{listings}
	\usepackage{color}
	\usepackage{amsmath}
	\definecolor{lightgray}{rgb}{.9,.9,.9}
	\definecolor{darkgray}{rgb}{.4,.4,.4}
	\definecolor{purple}{rgb}{0.65, 0.12, 0.82}
	% Folgende Angaben bitte anpassen
	\renewcommand{\Vorlesung}{VIS}
	\renewcommand{\Semester}{WiSe 2017}
	\renewcommand{\Aufgabenblatt}{7}
	\renewcommand{\Teilnehmer}{Hennings, Regorz, Röder, Budde, Warrelmann}
	\begin{document}
	\setcounter{section}{0}

	\aufgabe{Kryptografie}
	
	\subsection*{a)}
	Der Angreifer könnte relativ einfach mit einer known-plaintext-attack die Schlüssel $K_{1}$ und $K_{2}$ heraus finden.
	Hierbei wird der bekannte Klartext M mit allen möglichen Belegungen für $K_{1}$ verschlüsselt. Beide Werte für $K_{1}$ und v werden in einer Tabelle gespeichert $[[\{value: k_{1i}\} ,\{v_{i}\}], ... ]$ Das selbe geschieht für die bekannte Chiffre C aus M. Diese wird wiederum mit allen möglichen Belegungen für $K_{2}$ entschlüsselt und auch hier werden die Werte in eine Tabelle geschrieben $[[\{value: k_{2j}\} ,\{w_{j}\}], ... ]$ . Gillt nun $v_{i} == w_{j}$ für ein bestimmtes Paar i und j, dann sind $K_{1i}$ und $K_{2j}$ die gesuchten Schlüssel.\\\
	Der Auffwand beträgt:
	$2^{56}+2^{56} =2\cdot 2^{56} =2^{57}$\\\
	–  Sicherheitsgewinn wäre nur 1 Bit
	\\\\\\
	 \textbf{Wir sollen diese Sicherheit mit der Vergleichen, die durch zwei 56 bit Ks erreicht wird, aber genau das tun wir ja hier. Weis jemand von euch was die wollen??}
		
	\subsection*{b)}
	Die effektive Sicherheit eines 3-DES mit 3 Schlüsseln:\\\
	$2^{56}+2^{56}+2^{56} =2\cdot 2\cdot 2^{56} =2^{58}$\\\
	–  Sicherheitsgewinn wären somit zwei Bit, was im Gegensatz zum Sicherheitsgewinn von DES zu DES-2 signifikant mehr ist, da es sich hier um exponentielles Wachstum handelt. Der 3-DES mit 3 Schlüsseln ist doppelt so sicher wie
	 2-DES.
	 	\\\\\\
	 \textbf{Der Aufgabenteil ist nur angefangen und mit Unsicherheiten, kann sich da noch wer dran setzen?}
		
    \aufgabe{Unsicherheit des Electronic-Codebook-Modus}
    
	\aufgabe{Cipher-Block-Chaining-Modus}
	
	\subsection*{a)}
	Zwei identische Klartexte die lediglich einen anderen Initialisierungsvektor haben sind nur in der Länge identisch.
	Der Vektor wird genutzt um eine vom Schlüssel unabhängige randomisierung zu erreichen.
	
	\subsection*{b)}
	Auch hier unterscheidet sich das Chiffrat stark voneinader, da Folgeblöcke vom Vorgänger abhängen. Ändert sich also ein Bit im ersten Block, hat das Auswirkungen auf alle folgenden.
	
	\subsection*{c)}
	Eine Änderung eines Bits in einem Chiffratblock hat zur Folge, dass der entsprechende Nachrichtenblock randomisiert wird, während im Klartext des darauffolgenden Blocks genau die gleichen Bits geändert werden.

	\subsection*{d)}
	Die entschlüsselung wäre nur möglich, wenn der Initialisierungsverktor bekannt ist. Diesen könnte man dann mit sämtlichen möglichen Belegungen durchtesten und die Ergebnisse Vergleichen.
	
	\aufgabe{Hybrides Kryptosystem}
	
	\aufgabe{Das Diffi-Hellmann-Schlüsselaustauschprotokoll}
	
	\aufgabe{Sicherheit des RSA Verfahrens}
	
	\subsection*{a)}
	
	Das RSA-Verfahren basiert auf der Faktorisierungsannhame, die besagt, dass es nicht effizient möglich ist, die Primfaktoren einer großen Zahl zu berechnen. \\
	Ein Angreifer besitzt den öffentlichen Schlüssel, der beispielsweise die Werte $c=7$ und $n=299$ haben kann, und die verschlüsselte Nachricht, die beispielsweise $c(m)=107$ lauten kann. Aufgrund der Faktorisierungsannahme, kann der Angreifer nicht die beiden Primfaktoren $p=13$ und $q=23$ von $n$ bestimmen. Dadurch ist es ihm nicht möglich die Zahl $\phi(n)=(p-1)\cdot(q-1)=(13-1)\cdot(23-1)=264$ zu bestimmen, die er benötigen würde, um mit Hilfe des erweiterten euklidischen Algorithmus das multiplikative Inverse $d$ von $c$ in $\phi(n)$ mit $d=151$ zu errechnen. Da der Angreifer also aufgrund der Faktorisierungsannahme mit diesem Verfahren nicht an das $d$ kommen wird, wird er nie herausfinden, dass der Inhalt der Nachricht $c(m)^d \mod n = 107^{151} \mod 299 = 42$ lautet.
	
	\subsection*{b)}
	Sei zunächst $m_k$ das $k$-te Zeichen des Textes, $c$ der Exponent aus der zweiten Textdatei sowie $n$ der zugehörige Modulus. Dann enthält die erste Textdatei die Einträge $e_k = m_k^c \mod n$. \\
	\\
	Die erste Möglichkeit besteht darin, für ein beliebiges $k$ das $m\in\{0,...,256\}$ mit $m^c\mod n=e_k$ durch Ausprobieren zu bestimmen. Dann lässt sich durch iterieren über $i\in\{0,...\}$ das $i$ mit $e_k^i \mod n = m$ bestimmen. Dieses $i$ ist das $d$ des privaten Schlüssels, mit dem sich der Rest der Textdatei entschlüsseln lässt. \\
	\\
	Die zweite Möglichkeit besteht darin, ein Dictionary $f$ mit $f(m^c \mod n) = m\ \forall m\in \{0,...,256\}$ zu erstellen und mithilfe dieses Dictionaries die Datei zu entschlüsseln. Diese Möglichkeit wurde im folgenden Racket-Programm inplemeniert:
	\begin{lstlisting}
#lang racket
; ############# Loading Data ################################
	
(define schluesseltexte (file->list "ciphertext.txt"))

(define publickey (car (file->list "public_key.txt")))
(define modulus (cadr (file->list "public_key.txt")))

; ############# Creating Dictionany ##########################

(define klartextliste (range 0 256))

(define dict (map
	(lambda (klartext)
		(cons klartext
			(remainder (expt klartext publickey) modulus)))
	klartextliste))

; ############## Do attack ###################################

(define (finde-klartext schluesseltext)
	(car (findf (lambda (entry)
			(= schluesseltext (cdr entry)))
			dict)))

(list->string (map (compose1
		integer->char
		finde-klartext)
	schluesseltexte))
	\end{lstlisting}
Der berechnete Klartext lautet "Fuer die VIS-Klausur sind alle Inhalte, die in Uebung und Vorlesung behandelt wurden wichtig. Viel Erfolg! :-)"
	\end{document}
