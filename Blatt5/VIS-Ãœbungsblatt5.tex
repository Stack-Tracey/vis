\documentclass[ngerman]{fbi-aufgabenblatt}
%	\usepackage{tikz}
	\usepackage{listings}
	\usepackage{color}
	\definecolor{lightgray}{rgb}{.9,.9,.9}
	\definecolor{darkgray}{rgb}{.4,.4,.4}
	\definecolor{purple}{rgb}{0.65, 0.12, 0.82}
	% Folgende Angaben bitte anpassen
	\renewcommand{\Vorlesung}{VIS}
	\renewcommand{\Semester}{WiSe 2017}
	\renewcommand{\Aufgabenblatt}{5}
	\renewcommand{\Teilnehmer}{Hennings, Regorz, Röder, Budde, Warrelmann}
	\begin{document}
	\setcounter{section}{0}

	\aufgabe{Domain Name System}
	
	
	\subsection*{a)}
	
	\subsection*{c)}

  Der Befehl \code{dig www.inf.uni-hamburg.de. @213.73.91.35}
  lieferte folgende IP-Adresse: 134.100.56.130.
  \newline

	\subsection*{d)}

	\subsection*{e)}

	\aufgabe{Remote Method Invocation}
	
	\subsection*{b)}
  Der Server erstellt und exportiert eine RMI-Registry, die für einen bestimmten 
  Port request akzeptiert. Bei der Registry wird auch das Remote Objekt
  angemeldet.  Der Client holt sich über die RMI-Registry die Referenz des
  Remote Objektes vom Server. Nach erhalt der Referenz zum Remote Objekt,
  kann der Client über ein Interface Methoden an seinem Remote Objekt aufrufen.
  Der Client erhält Rückgabewerte, sowie die zu beachtenden RemoteExceptions.
  Die JVM vom Server führt die Methoden auf dem Objekt aus.

	\subsection*{c)}


\end{document}
